\documentclass{elsarticle}

\begin{document}

\title{The OpenMC Monte Carlo Code\tnoteref{t1}}
\tnotetext[t1]{This research was performed under appointment to the Rickover
  Fellowship Program in Nuclear Engineering sponsored by Naval Reactor Division
  of the U.S. Department of Energy.}
\author[mit]{P.K. Romano\corref{cor1}}
\ead{paul.k.romano@gmail.com}
\cortext[cor1]{Corresponding author}

\author[mit]{B. Forget}
\ead{bforget@mit.edu}

\author[mit]{K. Smith}
\ead{kord@mit.edu}

\address[mit]{Massachusetts Institute of Technology, Department of Nuclear
  Science and Engineering, 77 Massachusetts Avenue, Building 24-213, Cambridge,
  MA 02141}

\begin{abstract}
Insert abstract here
\end{abstract}

\maketitle

\section{Introduction}

The introduction of exascale computing in the next decade will introduce a
variety of challenges both for hardware and software developers. As such,
research and development efforts are underway aimed at enabling high-fidelity,
large-scale simulations that will scale on current and future computer
achitectures. To support these studies, a new Monte Carlo code has been under
development since early 2011 at the Massachusetts Institute of Technology. The
goal in developing a new Monte Carlo code rather than using a previously
developed code is to have a code that is easily extensible for research purposes
in addition to being high performance, freely available, and written in a modern
programming language.

\section{Methods}

\subsection{Physics}

The initial work on OpenMC has focused on criticality calculations as applied to
the simluation of nuclear reactors. The solution of the eigenvalue problem
proceeds by the method of successive generations [REF] wherein a constant number
of neutron histories are tracked from birth to death. The data governing the
interaction of neutrons with various nuclei are represented using the ACE format
[REF] which is used by MCNP \cite{mcnp} and Serpent \cite{serpent}. ACE-format
data can be generated with the NJOY nuclear data processing system which
converts raw ENDF/B data into linearly-interpolable data as required by most
Monte Carlo codes. The use of a standard cross section format allows for a
direct comparison of OpenMC with other codes since the same cross section
libraries can be used.

The ACE-format contains continuous-energy cross sections for the following types
of reactions: elastic scattering, fission (or first-chance fission,
second-chance fission, etc.), inelastic scattering, (n,xn), (n,$\gamma$), and
various other absorption reactions. For those reactions with one or multiple
neutrons. For those reactions with one or more neutrons in the exit channel,
secondary angle and energy distributions may be present. In addition,
fissionable nuclides have total, prompt, and/or delayed $\nu$ as a function of
energy, neutron precursor distributions. Many nuclides also have probability
tables to be used for accurate treatment of self-shielding in the unresolved
resonance range. For bound scatterers, separate tables with $S(\alpha,\beta)$
scattering law data can be used. Cross sections are represented as tabulated
functions of energy that are linearly interpolated between sucessive values.

For neutrons at higher energies, it can be safely assumed that the motion of the
target nucleus is negligible relative to the velocity of the neutron
itself. However, in the thermal and intermediate energy ranges, the target
velocity will alter both the cross sections and the secondary energy and angle
distributions of scattered neutrons. To account for the effect on cross
sections, Doppler broadening is typically performed in the cross section
generation stage. For the angle and energy distributions, OpenMC uses a free gas
approximation \cite{freegas} wherein the velocity of the target nuclei have a
Maxwellian distribution. For thermal neutrons scattering from bound molecules
such as hydrogen or deuterium in water, graphite, beryllium, etc., the free gas
approximation will not accurately capture the scattering kinematics and
$S(\alpha,\beta)$ scattering law data must be used.

In the unresolved resonance energy range, it is not adequate to use smooth cross
sections since the experimental resolution is not fine enough to resolve all
resonances. To properly account for self-shielding in this energy range, OpenMC
uses the probability table method \cite{probtables}. For most thermal and fast
reactors, the use of probability tables will not significantly affect problem
results. However, for some problems with an appreciation flux spectrum in the
unresolved resonance range, not using probability tables will lead to incorrect
results \cite{probtables-testing}.

\subsection{Geometry}

Geometric objects are modeled in OpenMC using constructive solid geometry
whereby any closed volume is represented as the union or intersection of one or
multiple second-order surfaces. Transmitting, vacuum, or reflective boundary
conditions can be applied to any surface. 

\subsection{Tallies}

Description of tally system.

\subsection{Input and Files}

Description of XML files.

\subsection{Parallelism}

Description of fission bank method.

\subsection{Code Development}

Modern fortran 2003, dictionaries/lists, git/github for version control, issue
tracking, etc., Sphinx for documentation. Support for multiple compilers at
varying levels of optimization and compilation flags.

\section{Results}

\subsection{Benchmarks}

\subsection{Parallel Scaling}

\section{Conclusions}

\bibliography{main}
\bibliographystyle{elsarticle-num}

\end{document}
